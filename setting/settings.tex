% SETTINGS

% Font encoding
\usepackage[T1]{fontenc}

% Frontespizio definitions
\usepackage[standard,norules]{frontespizio}

% Graphics
\usepackage{graphicx}
\usepackage{animate} % animated gifs
\DeclareGraphicsExtensions{.pdf,.png,.jpg}
\graphicspath{{img/}}

% Hyphenation
\usepackage[english]{babel}

% Colors and frames
\usepackage{framed}
\usepackage{color}

% Enchant math and other fonts
\usepackage{textcomp}
\usepackage[nointegrals]{wasysym}
\usepackage{mathtools}

% Hyperref settings
\hypersetup{%
 	pdftitle={Autonomous VTOL for avalanche buried searching: Avionics},%     % title
  pdfauthor={Matteo Ragni},%                                                % author
  pdfsubject={Master Thesis in Mechatronics Engineering},%                  % subject of the document
  pdfcreator={Matteo Ragni -- pdflatex},%                                   % creator of the document
  pdfproducer={University of Trento, Department of Industrial Engineering}% % producer of the document
}
% minitoc in starting of the chapter
\usepackage{minitoc}
\setcounter{tocdepth}{1}
\setcounter{minitocdepth}{4}
\setcounter{secnumdepth}{3}

% International system units
\usepackage{siunitx}

% Table of symbols
\usepackage[final]{listofsymbols} % To use for final version
%\usepackage{listofsymbols} % To use during editing

% insert algorithm
\usepackage[ruled,vlined,linesnumbered]{algorithm2e}
\newcommand{\commentstyle}[1]{\footnotesize\ttfamily\textcolor{gray}{#1}}
\SetCommentSty{commentstyle}

% Insert href in text
\usepackage{url}

% Rotate text
\usepackage{rotating}

% composition of table
\usepackage{array}
\usepackage{booktabs}

%% Sectioning formatting
%% CHAPTER DEFINITION
\usepackage[Bjornstrup]{fncychap}

%\usepackage{titlesec} Already loaded by tufte-book
%% SECTION DEFINITION
\definecolor{lgray}{gray}{0.843}
\titleformat{\section}[hang]{\Large\bfseries}{}{0em}{%
  {%
    \setlength{\fboxsep}{0pt}%
    \colorbox{lgray}{\makebox[\textwidth]{\Large\strut}}%
  }%
  \hspace*{-\textwidth}%
  \hspace*{1em}%
  \thesection%
  \hspace*{1em}%
}[]
%% SUBSECTION DEFINITION
\titleformat{\subsection}[hang]{\bfseries}{}{0em}{%
  {%
    \setlength{\fboxsep}{0pt}%
    \colorbox{lgray}{\makebox[\textwidth]{\strut}}%
  }%
  \hspace*{-\textwidth}%
  \hspace*{1em}%
  \thesubsection%
  \hspace*{1em}%
}[]

%% PARAGRAPH DEFINITION
\newcommand{\myparagraph}[1]{\paragraph{#1} \leavevmode \vspace{0.2em} \\}

%% Old versions for sectioning format
% \titleformat{\chapter}[frame]
% {\normalfont}{\filright \enspace \sffamily \thechapter\enspace}{8pt}
% {\Large\bfseries\filcenter}

%\titleformat{\section}[block]
%{\normalfont}{\thesection}{6pt}{\large\bfseries}

%\titleformat{\subsection}[block]
%{\normalfont}{\thesubsection}{6pt}{\bfseries}



% ANIMATIONS
% convert immagine.gif -coalesce animation_%d.pdf
% poi inserire nel latex
% \animategraphics[height=%DIMENSIONE%,autoplay,loop]{%POSIZIONE_ANIMAZIONE%/animation_}


%% REMOVE IDENTATION
\usepackage{parskip}

\makeatletter
% Paragraph indentation and separation for normal text
\renewcommand{\@tufte@reset@par}{%
  \setlength{\RaggedRightParindent}{0.0pc}%
  \setlength{\JustifyingParindent}{0.0pc}%
  \setlength{\parindent}{0pc}%
  \setlength{\parskip}{3pt}%
}
\@tufte@reset@par

% Paragraph indentation and separation for marginal text
\renewcommand{\@tufte@margin@par}{%
  \setlength{\RaggedRightParindent}{0.0pc}%
  \setlength{\JustifyingParindent}{0.0pc}%
  \setlength{\parindent}{0.0pc}%
  \setlength{\parskip}{3pt}%
}
\makeatother

\usepackage[titles]{tocloft}
\setlength{\cftsecindent}{0cm}
\setlength{\cftsecnumwidth}{25pt}
\setlength{\cftsubsecindent}{\cftsecindent}
\addtolength{\cftsubsecindent}{\cftsecnumwidth}
\setlength{\cftsubsecnumwidth}{25pt}
\setlength{\cftsubsubsecindent}{0cm}
\addtolength{\cftsubsubsecindent}{0cm}
\setlength{\cftparaindent}{50pt}
\addtolength{\cftparaindent}{0cm}
\renewcommand{\cftparafont}{\it}
\tocloftpagestyle{empty}

% Header and footer style
%\usepackage{fancyhdr}
%\pagestyle{fancy}
%\renewcommand{\chaptermark}[1]%
%{\markboth{{\thechapter.\ #1}}{}}
%\renewcommand{\sectionmark}[1]%
%{\markright{{\thesection.\ #1}}}
%\renewcommand{\headrulewidth}{0.1pt}
%\renewcommand{\footrulewidth}{0pt}
%\fancyhead[LE,RO]{\thepage}
%\fancyhead[LO]{\rightmark}
%\fancyhead[RE]{\leftmark}

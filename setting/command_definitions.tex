% COMMAND DEFINITIONS

% Definition of the abstract environment

\newenvironment{abstract}{%
	\cleardoublepage \thispagestyle{empty} \begin{fullwidth} \null \vspace{\stretch{1}} \begin{center}%
	\bfseries \colorbox{lgray}{{ } \hspace{0.438\linewidth} \abstractname \hspace{0.438\linewidth} { }} \end{center}}{%
	\vspace{\stretch{1}} \null \end{fullwidth} \clearpage%
}

\newenvironment{aknowledgements}{%
	\cleardoublepage \thispagestyle{empty} \begin{fullwidth} \null \vspace{\stretch{4}} \begin{center}%
	\bfseries \end{center}}{%
	\vspace{\stretch{1}} \null \end{fullwidth} \clearpage %
}

\newenvironment{dedication}{%
	\thispagestyle{empty}%
	\begin{fullwidth}%
	\begin{flushright}%
	\null \vspace{\stretch{1}}}{%
	\vspace{\stretch{2}} \null%
	\end{flushright} \end{fullwidth} \clearpage%
}

\newcommand{\listofsymbolspages}{
	\cleardoublepage%
	\chapter*{List of symbols}%
	\addcontentsline{toc}{chapter}{List of symbols}%
	\listofsymbols%
}

\newcommand{\bibliographyinsert}{%
	\bibliography{bib/bibs}%
	\bibliographystyle{plain}%
}

\def\inch{''}
\def\euro{\texteuro}

%% MATH DEFINITION
% d/dt
\def\partialt{\dfrac{\partial}{\partial t}}
\newcommand{\partialtarg}[1]{\dfrac{\partial {#1}}{\partial t}}
\newcommand{\partialttarg}[1]{\dfrac{\partial^2 {#1}}{\partial t^2}}

% versors definition
\newcommand{\vers}[1]{\hat{\mathbf{#1}}}
\def\vrx{\vers{x}}
\def\vry{\vers{y}}
\def\vrz{\vers{z}}
\def\vrr{\vers{r}}
\def\vrtheta{\hat{\boldsymbol{\theta}}}
\def\vrphi{\hat{\boldsymbol{\phi}}}
\newcommand{\ccos}[1]{\cos\left( #1 \right)}
\newcommand{\ssin}[1]{\sin\left( #1 \right)}
\newcommand{\atan}[1]{\arctan\left( #1 \right)}
\newcommand{\braces}[1]{\left( #1 \right)}
\def\ritardotempo{\omegaarva \left( t - \dfrac{r}{\velocitaluce}\right)}
\def\ritardotempolinea{\omegaarva \left( t - r/\velocitaluce\right)}
\newcommand{\abs}[1]{\left| #1 \right|}
\def\magfieldmatrix{\left[\begin{array}{ccc}%
2x^2-y^2-z^2 & 3xy & 3xz \\%
3xy & 2y^2-x^2-z^2 & 3yz \\%
3xz & 3yz & 2z^2-x^2-y^2%
\end{array}\right]}
\newcommand{\naming}[2]{\underset{#2}{\underbrace{#1}}}

\newcommand{\arraymath}[1]{%
\[%
\begin{array}{rcl}#1\end{array}%
\]}
\newcommand{\vettore}[1]{%
\left[\begin{array}{c}#1\end{array}\right]%
}

% Symbol heading
%\renewcommand{\symheadingname}{}
\renewcommand{\symheading}{}

% TODO definition
\newcommand{\TODO}[1]{%
  \colorbox{yellow}{#1}%
}

\newcommand{\drawcube}[9]{%
	\coordinate (#7_A) at (#1,#2,#3);%
	\coordinate (#7_B) at (#1+#4,#2,#3);%
	\coordinate (#7_C) at (#1+#4,#2+#5,#3);%
	\coordinate (#7_D) at (#1,#2+#5,#3);%
	\coordinate (#7_E) at (#1+#4,#2,0);%
	\coordinate (#7_F) at (#1+#4,#2+#5,0);%
	\coordinate (#7_G) at (#1,#2+#5,0);%
	%
	\filldraw [draw=black, fill=#6] (#7_A) -- (#7_B) -- (#7_C) -- (#7_D) -- cycle;%
	\filldraw [draw=black, fill=#6] (#7_B) -- (#7_C) -- (#7_F) -- (#7_E) -- cycle;%
	\filldraw [draw=black, fill=#6] (#7_D) -- (#7_C) -- (#7_F) -- (#7_G) -- cycle;%
	\node [#9] at (0.5*#4+#1,0.5*#5+#2,#3) {#8};%
}

\newcommand{\drawplanezy}[8]{%
	\coordinate (#5_A) at (#1,#2,#3);%
	\coordinate (#5_B) at (#1,#2,0);%
	\coordinate (#5_C) at (#1,#2+#4,0);%
	\coordinate (#5_D) at (#1,#2+#4,#3);%
	\draw [#6] (#5_A) -- (#5_B) -- (#5_C) -- (#5_D) -- cycle;%
	\node [#8] at (#1,#2+#4-0.4,0.5*#3) {#7};%
}

\newcommand{\drawplanexy}[7]{%
	\coordinate (#7_A) at (#1,#2,#3);%
	\coordinate (#7_B) at (#1+#4,#2,#3);%
	\coordinate (#7_C) at (#1+#4,#2+#5,#3);%
	\coordinate (#7_D) at (#1,#2+#5,#3);%
	\draw [#6] (#7_A) -- (#7_B) -- (#7_C) -- (#7_D) -- cycle;%
}

\newcommand{\fromworkspace}[3]{%
	\node [block, draw=white,#3, minimum height=1.5em] (#1) {#2};%
	\coordinate [at=(#1.east), xshift=5] (#1_out);%
	\draw (#1.north west) -- (#1.north east) -- (#1_out) -- (#1.south east) -- (#1.south west) -- cycle;%
}
\newcommand{\toworkspace}[3]{%
	\node [block, draw=white,#3] (#1) {#2};%
	\coordinate [at=(#1.west), xshift=-5] (#1_in);%
	\draw (#1.north west) -- (#1.north east) --  (#1.south east) -- (#1.south west) -- (#1_in) -- cycle;%
}
\newcommand{\inputpin}[3]{%
	\node [circle, draw, text width=0.5cm,align=center,#3] (#1) {#2};%
	\coordinate [at=(#1.center), xshift=0.75cm] (#1_out);%
	\draw (#1.north east) -- (#1_out) -- (#1.south east);%
}
\newcommand{\outputpin}[3]{%
	\node [circle, draw, text width=0.5cm,align=center,#3] (#1) {#2};%
	\coordinate [at=(#1.center), xshift=-0.75cm] (#1_in);%
	\draw (#1.north west) -- (#1_in) -- (#1.south west);%
}
% COMMAND DEFINITIONS

% Definition of the abstract environment

\newenvironment{abstract}{%
\cleardoublepage \thispagestyle{empty} \null \vfill \begin{center}%
\bfseries \abstractname \end{center}}{%
\vfill \null%
}

\newenvironment{aknowledgements}{%
\cleardoublepage \thispagestyle{empty} \null \vspace{\stretch{4}} \begin{center}%
\bfseries \end{center}}{%
\vspace{\stretch{1}} \null%
}

\newenvironment{dedication}{%
\thispagestyle{empty}%
\begin{flushright}%
\null \vspace{\stretch{1}}}{%
\vspace{\stretch{2}} \null
\end{flushright}}

\newcommand{\listofsymbolspages}{%
\thispagestyle{empty}
\cleardoublepage%
\chapter*{List of symbols}
\addcontentsline{toc}{chapter}{List of symbols}
%\addtocontents{toc}{\protect\contentsline{section}{List of Symbols}{}}%
\listofsymbols}

\newcommand{\bibliographyinsert}{%
\bibliography{bib/bibs}%
\bibliographystyle{plain}%
}

\newcommand{\myparagraph}[1]{\paragraph{#1} \leavevmode \vspace{0.2em} \\}

\def\inch{''}
\def\euro{\texteuro}

%% MATH DEFINITION
% d/dt
\def\partialt{\dfrac{\partial}{\partial t}}
\newcommand{\partialtarg}[1]{\dfrac{\partial {#1}}{\partial t}}
\newcommand{\partialttarg}[1]{\dfrac{\partial^2 {#1}}{\partial t^2}}

% versors definition
\newcommand{\vers}[1]{\hat{\mathbf{#1}}}
\def\vrx{\vers{x}}
\def\vry{\vers{y}}
\def\vrz{\vers{z}}
\def\vrr{\vers{r}}
\def\vrtheta{\hat{\boldsymbol{\theta}}}
\def\vrphi{\hat{\boldsymbol{\phi}}}
\newcommand{\ccos}[1]{\cos\left( #1 \right)}
\newcommand{\ssin}[1]{\sin\left( #1 \right)}
\newcommand{\atan}[1]{\arctan\left( #1 \right)}
\newcommand{\braces}[1]{\left( #1 \right)}
\def\ritardotempo{\omegaarva \left( t - \dfrac{r}{\velocitaluce}\right)}
\def\ritardotempolinea{\omegaarva \left( t - r/\velocitaluce\right)}
\newcommand{\abs}[1]{\left| #1 \right|}
\def\magfieldmatrix{\left[\begin{array}{ccc}%
2x^2-y^2-z^2 & 3xy & 3xz \\%
3xy & 2y^2-x^2-z^2 & 3yz \\%
3xz & 3yz & 2z^2-x^2-y^2%
\end{array}\right]}


\newcommand{\arraymath}[1]{%
\[%
\begin{array}{rcl}#1\end{array}%
\]}

% Symbol heading
\renewcommand{\symheadingname}{}

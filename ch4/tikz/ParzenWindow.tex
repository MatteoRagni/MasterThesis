\documentclass{standalone}
\usepackage{tikz}
\usepackage{tikz-3dplot}

\usepackage{mathtools}
\usepackage{pgfplots}

\usetikzlibrary{shapes,arrows,matrix,fit,positioning,decorations.pathreplacing,3d,calc}
\usepgfplotslibrary{groupplots}
\usepackage{pgfplotstable}
\usepackage{filecontents}

\pgfplotsset{compat=1.11}
\usepackage{mathpazo} 

\begin{document}
\begin{tikzpicture}
	\begin{axis}[
		name=plot1, height=12cm, width=12cm,
		xmin=-30, xmax=60, ymin=-60, ymax=30,
		grid style={color=gray!40, dotted}, grid=major,
		xlabel=$x \mathrm{(m)}$, ylabel=$y \mathrm{(m)}$,
		tick label style={font=\scriptsize}
	]
		\addplot [mark=none] table[x index=0, y index=1] {data/position_projection.txt};	
		\addplot[only marks, mark=otimes*,fill=red,color=red,mark size=2pt] table [x index=0, y index=1] {data/transmitter_pos.txt};
		\addplot[only marks, mark=x,color=gray!75,mark size=1pt] table {data/optimizer_data.txt};
		\addplot[only marks, mark=*,color=blue,mark size=0.75pt] table {data/obstacle_pos.txt};
	\end{axis}

	\begin{axis}[
		name=plot2, at={($(plot1.east)+(4.2cm,0)$)}, anchor=east, 
		height=12cm, width=5cm,
		ymin=-60, ymax=30, xmin=0, xmax=0.15,
		grid style={color=gray!40, dotted}, grid=major,
		tick label style={font=\scriptsize},
		xlabel=$p(\mathbf{p}_{\mathrm{tx},y}|\mathbf{H})$,
		xticklabel style={ /pgf/number format/precision=2,/pgf/number format/fixed,/pgf/number format/fixed zerofill }
	]
		\addplot [mark=none, fill=gray!40] table[x index=3, y index=2] {data/parzenWin_data.txt};
	\end{axis}
	
	\begin{axis}[
		name=plot3, at={($(plot1.north)+(0,4.2cm)$)}, anchor=north, 
		height=5cm, width=12cm,
		ymin=0, ymax=0.15, xmin=-30, xmax=60,
		grid style={color=gray!40, dotted}, grid=major,
		tick label style={font=\scriptsize},
		ylabel=$p(\mathbf{p}_{\mathrm{tx},x}|\mathbf{H})$,
		yticklabel style={ /pgf/number format/precision=2,/pgf/number format/fixed,/pgf/number format/fixed zerofill }
	]
		\addplot [mark=none, fill=gray!40] table[x index=0, y index=1] {data/parzenWin_data.txt};
	\end{axis}

\end{tikzpicture}
\end{document}

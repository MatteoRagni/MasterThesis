% CHAPTER 2

\begin{fullwidth}
\chapter{Design of a digital ARTVA \label{ch:chapter2}}
\end{fullwidth}
\minitoc

In this chapter we will try to design a digital receiver for the ARTVA signal. Before starting with the design, \emph{ARTVA signal} is analyzed in detail, with the derivation of a simplified model of field pattern that could be used  for the implementation of the searching algorithm. After, ferrite antennas are analyzed, as they are the only way to receive such a long wavelength. In the last part of the chapter the circuitry for the ARTVA receiver is shown.

\section{Analysis of transmitting pattern}

A formal model of the transmitting pattern is fundamental for the implementation of the searching algorithm. We start from the basic Maxwell equation and we arrive to a simpler model numerically usable.

As we will see, radiating pattern is quite complex due to the fact that we are working in the \textbf{near--field} region, condition that constraint not to use classical DoA\marginnote{\emph{DoA}: Direction of Arrival}, such as MUSIC or ESPRIT, that operates in far-field condition and at higher frequencies. DoA systems for long waves usually requires too big electro--mechanical devices.

\subsection{Maxwell's Equations}
The following investigation is based upon Maxwell's Equation, in which the magnetic permeability $\magperm$ and $\dielettrico$ is considered as a constant (the radiation is assumed to propagate at speed of light in air). Also we consider some field properties, that are function of radio distance and time:
\[
f(\radiodist,t) = f \qquad f \in \left[\; \chargedens,\; \currdens,\; \efield,\; \bfield,\; \hfield \; \right]
\]
The equations that rule the induction are the \emph{Gauss equation of magnetic induction} and the \emph{Faraday law of electric induction}:
\begin{equation}
\nabla\cdot\bfield = 0
\label{eq:gauss1}
\end{equation}
\begin{equation}
\nabla\times\efield = -\partialt \bfield
\label{eq:faraday}
\end{equation}
while the equations that rule the interaction with material are \emph{Gauss equation} and \emph{Ampere law}:
\begin{equation}
\nabla\cdot\efield = \dfrac{\chargedens}{{\dielettrico}_0}
\label{eq:gauss2}
\end{equation}
\begin{equation}
\nabla\times\bfield = {\magperm}_0 \left( \currdens + {\dielettrico}_0 \partialt \efield \right)
\label{eq:ampere}
\end{equation}

\subsection{EM field dynamic potentials}

Starting from equation \ref{eq:gauss1}, we can define a vectorial function called \emph{potential vector} $\afield$ of $\bfield$:
\begin{equation}
\bfield = \nabla\times\afield
\label{eq:potvett}
\end{equation}
\marginnote{The existence od $\afield$ is verified by property of $\nabla$ operator, whom states that the divergence of a curl of a vector field is zero} Putting \ref{eq:potvett} in \ref{eq:faraday}:
\[
\nabla\times\efield = - \partialt \left( \nabla\times\afield \right)
\]
\begin{equation}
\nabla\times\left( \efield + \partialtarg{\afield} \right) = 0
\end{equation}
From the previous equation it is evident that the argument between parenthesis is in reality an irrotational vector field, thus a potential function exists such that:
\[
-\nabla\scpot = \efield + \partialtarg{\afield}
\]
and we derive the following definition of electric field:
\begin{equation}
\efield = - \nabla \scpot - \partialtarg{\afield}
\label{eq:efieldvett}
\end{equation}
Equation \ref{eq:potvett} and \ref{eq:efieldvett} are used to express a new formulation for the Maxwell's equation based upon vector potential\sidenote{The proof is in chapter appendix \ref{eq:evidence1}}:
\begin{equation}
\begin{array}{rcl}
\nabla^2\scpot + \partialt \nabla \cdot \afield & = & - \dfrac{\chargedens}{\dielettrico_0} \\
\nabla^2 \afield - \dfrac{1}{\velocitaluce^2} \partialttarg{\afield} - \nabla\left( \nabla \cdot \afield + \dfrac{1}{\velocitaluce^2} \partialtarg{\scpot} \right) & = & -\magperm_0 \currdens
\end{array}
\label{eq:potvecmaxwell}
\end{equation}
Those equation, even if the evident complexity, could be resolved as a well posed boundaries condition problem. Equation are coupled with the current formulation, but could be decoupled using the \textbf{Gauge transformation}



%%%%%%%%%%%%%%%%%%%%%%%%%%%%%%%%%%%%%%%%%%%%%%%%%%%%%%%%%%%%%%%%%%%%%%%%%%%%%%%%%%%%

\section{Appendix}

\subsection{Evidence}

\myparagraph{EM field dynamic potential}
The following equations are the proof for \ref{eq:potvecmaxwell}
\begin{equation}
\begin{array}{rcl}
\nabla \cdot \left(  - \nabla \scpot - \partialtarg{\afield} \right) & = & \dfrac{\chargedens}{\dielettrico_0} \\
\nabla^2 \scpot + \partialt{\nabla \cdot \afield} & = & -\dfrac{\chargedens}{\dielettrico_0} \\
 & & \\
\nabla \times \left( \nabla \times \afield \right) & = & \magperm_0 \left( \currdens + \dielettrico_0 \partialtarg{\left(- \nabla \scpot - \partialtarg{\afield}\right)} \right) \\
\nabla \left( \nabla \cdot \afield \right) - \nabla^2 \afield & = & \magperm_0 \currdens - \magperm_0 \dielettrico_0 \dfrac{\partial^2 \afield}{\partial t^2} - \magperm_0 \dielettrico_0 \nabla \dfrac{\partial \scpot}{\partial t} \\
\nabla^2 \afield - \dfrac{1}{c^2} \dfrac{\partial^2 \afield}{\partial t^2} - \nabla \left( \nabla \cdot \afield + \dfrac{1}{c^2} \dfrac{\partial \scpot}{\partial t} \right) & = & - \magperm_0 \currdens
\end{array}
\label{eq:evidence1}
\end{equation}



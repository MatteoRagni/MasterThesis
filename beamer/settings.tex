\usepackage{siunitx}
\usepackage{array}
\usepackage{animate}
\usepackage{graphicx} 

% Algorithms
\usepackage[vlined]{algorithm2e}
\newcommand{\commentstyle}[1]{\footnotesize\ttfamily\textcolor{gray}{#1}}
\SetCommentSty{commentstyle}
\usepackage{rotating}

% TikZ
\usepackage{tikz}
\usepackage{tikz-3dplot}
\usepackage{pgfplots}
\usetikzlibrary{shapes,arrows,positioning,matrix,decorations.pathreplacing,3d,calc}


% Visualization geometry
\geometry{a5paper,landscape}
%\geometry{screen,centering}

% Theme settings
\usetheme{Frankfurt}
\usecolortheme{seagull}
\usefonttheme{serif}
\setbeamercolor{background canvas}{bg=white}
\setbeamerfont{section in toc}{size=\huge}

% Definition of the beamer numbered list style
\setbeamertemplate{section in toc}[default]
\setbeamertemplate{itemize items}[default]

% User settings
\title{Autonomous VTOL for Avalanche Buried Searching \\ {\Huge Avionics}}
\author{Matteo Ragni}
\date{Ingegneria Meccatronica Robotica}

% Settings toc
\setcounter{tocdepth}{1}

% Fonts settings
\usepackage{mathpazo}

\newcommand{\transitioneffectcommon}{\transfade[duration=0.25]}

\newcommand{\newslide}[2]{%
\subsection{#1}
\begin{frame}{\hspace{ \stretch{1} }\textbf{\huge#1}\hspace{ \stretch{1} }}
#2
\transitioneffectcommon
\end{frame}%
}

\newcommand{\newslidenn}[2]{%
\subsection*{#1}
\begin{frame}{\hspace{ \stretch{1} }\textbf{\huge#1}\hspace{ \stretch{1} }}
#2
\transitioneffectcommon
\end{frame}%
}

\def\bfield{\mathbf{B}}
\def\afield{\mathbf{A}}
\def\efield{\mathbf{E}}
\def\hfield{\mathbf{H}}
\def\currdens{\mathbf{J}}
\def\dielettrico{\varepsilon_0}
\def\magperm{\mu_0}
\def\partialt{\dfrac{\partial}{\partial t}}
\def\chargedens{\rho}
\renewcommand{\arraystretch}{1.85}

\newcommand{\ccos}[1]{\cos\left(#1\right)}
\newcommand{\ssin}[1]{\sin\left(#1\right)}
%\newcommand{\braces}[1]{\left(#1\right)}

% Diagrams styles
\tikzstyle{block} = [draw=black, rectangle, minimum height=2.5em, minimum width=2em]
\tikzstyle{sum} = [draw=black, circle]
\tikzstyle{gain} = [draw=black, regular polygon, regular polygon sides=3, regular polygon rotate=90, minimum height=2.5em, minimum width=2em]
\tikzstyle{input} = [coordinate]
\tikzstyle{output} = [coordinate]
\tikzstyle{pinstyle} = [pin edge={to-,thin,black}]

\newcommand{\drawcube}[9]{%
	\coordinate (#7_A) at (#1,#2,#3);%
	\coordinate (#7_B) at (#1+#4,#2,#3);%
	\coordinate (#7_C) at (#1+#4,#2+#5,#3);%
	\coordinate (#7_D) at (#1,#2+#5,#3);%
	\coordinate (#7_E) at (#1+#4,#2,0);%
	\coordinate (#7_F) at (#1+#4,#2+#5,0);%
	\coordinate (#7_G) at (#1,#2+#5,0);%
	%
	\filldraw [draw=black, fill=#6] (#7_A) -- (#7_B) -- (#7_C) -- (#7_D) -- cycle;%
	\filldraw [draw=black, fill=#6] (#7_B) -- (#7_C) -- (#7_F) -- (#7_E) -- cycle;%
	\filldraw [draw=black, fill=#6] (#7_D) -- (#7_C) -- (#7_F) -- (#7_G) -- cycle;%
	\node [#9] at (0.5*#4+#1,0.5*#5+#2,#3) {#8};%
}

\newcommand{\drawplanezy}[8]{%
	\coordinate (#5_A) at (#1,#2,#3);%
	\coordinate (#5_B) at (#1,#2,0);%
	\coordinate (#5_C) at (#1,#2+#4,0);%
	\coordinate (#5_D) at (#1,#2+#4,#3);%
	\draw [#6] (#5_A) -- (#5_B) -- (#5_C) -- (#5_D) -- cycle;%
	\node [#8] at (#1,#2+#4-0.4,0.5*#3) {#7};%
}

\newcommand{\drawplanexy}[7]{%
	\coordinate (#7_A) at (#1,#2,#3);%
	\coordinate (#7_B) at (#1+#4,#2,#3);%
	\coordinate (#7_C) at (#1+#4,#2+#5,#3);%
	\coordinate (#7_D) at (#1,#2+#5,#3);%
	\draw [#6] (#7_A) -- (#7_B) -- (#7_C) -- (#7_D) -- cycle;%
}

\newcommand{\fromworkspace}[3]{%
	\node [block, draw=white,#3, minimum height=1.5em] (#1) {#2};%
	\coordinate [at=(#1.east), xshift=5] (#1_out);%
	\draw (#1.north west) -- (#1.north east) -- (#1_out) -- (#1.south east) -- (#1.south west) -- cycle;%
}
\newcommand{\toworkspace}[3]{%
	\node [block, draw=white,#3] (#1) {#2};%
	\coordinate [at=(#1.west), xshift=-5] (#1_in);%
	\draw (#1.north west) -- (#1.north east) --  (#1.south east) -- (#1.south west) -- (#1_in) -- cycle;%
}
\newcommand{\inputpin}[3]{%
	\node [circle, draw, text width=0.5cm,align=center,#3] (#1) {#2};%
	\coordinate [at=(#1.center), xshift=0.75cm] (#1_out);%
	\draw (#1.north east) -- (#1_out) -- (#1.south east);%
}
\newcommand{\outputpin}[3]{%
	\node [circle, draw, text width=0.5cm,align=center,#3] (#1) {#2};%
	\coordinate [at=(#1.center), xshift=-0.75cm] (#1_in);%
	\draw (#1.north west) -- (#1_in) -- (#1.south west);%
}

\newcommand{\riferimento}{%
\draw[tdplot_rotated_coords,color=blue,thick,->] (0,0,0) -- (2,0,0) node[anchor=east]{$x’$};%
\draw[tdplot_rotated_coords,color=blue,thick,->] (0,0,0) -- (0,2,0) node[anchor=north]{$y’$};%
\draw[tdplot_rotated_coords,color=blue,thick,->] (0,0,0) -- (0,0,2) node[anchor=west]{$z’$};%
}

\newcommand{\braces}[1]{\left( #1 \right)}

\newcommand{\hcenter}[1]{ \hspace{\stretch{1}} \textbf{#1} \hspace{\stretch{1}} }
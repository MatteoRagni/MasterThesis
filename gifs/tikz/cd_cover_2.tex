\documentclass[tikz]{standalone}
\usepackage{graphicx}
\usetikzlibrary{calc}
\begin{document}

\newcommand{\stile}[1]{\textbf{#1}}
\newcommand{\studente}      {Davide Carpenito}
\newcommand{\corsolaurea}   {Ingegneria dei Materiali}
\newcommand{\titolotesi}    {Applicazione del metodo delle Breath Figures\\ per la preparazione di film microstrutturati\\ di polimeri fotoreticolati}
\newcommand{\relatori}      {Dott. Devid Maniglio}
\newcommand{\annoaccademico}{2013--14}
\newcommand{\datadiscus}    {22/10/2014}

\begin{tikzpicture}

	\draw [color=gray, dashed] (0cm,0cm) -- (12.05cm,0cm) -- (12.05cm,12.05cm) -- (0,12.05cm) -- cycle;
	\node at (6cm,10cm) {\includegraphics[width=9cm]{LogoUniTn.jpg}};

	\node at (1cm, 7cm) [anchor=180,align=left] {Corso di Laurea in:};
	\node at (1cm, 6cm) [anchor=180,align=left] {Laureando:};
	\node at (1cm, 4.5cm) [anchor=180,align=left] {Titolo della tesi:};
	\node at (1cm, 3cm) [anchor=180,align=left] {Relatore:};
	\node at (6cm, 1cm) [align=center] {\footnotesize{AA: \stile{\annoaccademico} -- Data Appello: \stile{\datadiscus}}};

	\node at (4.5cm, 7cm) [anchor=180,align=left,text width=6.5cm] {\stile{\corsolaurea}};
	\node at (4.5cm, 6cm) [anchor=180,align=left,text width=6.5cm] {\stile{\studente}};
	\node at (4.5cm, 4.5cm) [anchor=180,align=left,text width=6.5cm] {\stile{\titolotesi}};
	\node at (4.5cm, 3cm) [anchor=180,align=left,text width=6.5cm] {\stile{\relatori}};


\end{tikzpicture}

\end{document}